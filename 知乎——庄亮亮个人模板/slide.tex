\documentclass{beamer}
\usepackage{ctex, hyperref}
\usepackage{fontenc}

% other packages
\usepackage{latexsym,amsmath,xcolor,booktabs,calligra}
\usepackage{graphicx,pstricks,listings,stackengine}
\usepackage{multirow, multicol}
\author{庄闪闪}
\title{硕士学位论文答辩}
\subtitle{xxx统计分析}
\institute{xxx学院}
\date{2021年12月06日}
\usepackage{WZU}

% defs
\def\cmd#1{\texttt{\color{red}\footnotesize $\backslash$#1}}
\def\env#1{\texttt{\color{blue}\footnotesize #1}}
\definecolor{deepblue}{rgb}{0,0,0.5}
\definecolor{deepred}{rgb}{0.6,0,0}
\definecolor{deepgreen}{rgb}{0,0.5,0}
\definecolor{halfgray}{gray}{0.55}

\lstset{
    basicstyle=\ttfamily\small,
    keywordstyle=\bfseries\color{deepblue},
    emphstyle=\ttfamily\color{deepred},    % Custom highlighting style
    stringstyle=\color{deepgreen},
    numbers=left,
    numberstyle=\small\color{halfgray},
    rulesepcolor=\color{red!20!green!20!blue!20},
    frame=shadowbox,
}


\begin{document}

\kaishu
\begin{frame}
    \titlepage
    \begin{figure}[htpb]
        \begin{center}
            \includegraphics[width=0.2\linewidth]{pic/wzu_logo.png}
        \end{center}
    \end{figure}
\end{frame}

\begin{frame}
    \tableofcontents[sectionstyle=show,subsectionstyle=show/shaded/hide,subsubsectionstyle=show/shaded/hide]
\end{frame}


\section{背景介绍和研究现状}







\section{带区组效应重截尾区间数据的统计分析}



\subsection{构建模型}


\subsection{统计推断}

\subsubsection{二阶段方法}


\subsubsection{区间估计}

\begin{frame}{区间估计}
\begin{enumerate}
	\item \textbf{传统自助法}
	
	\textbf{缺点}xxx
	
	\item \textbf{分数随机加权自助法}
	
	\textbf{优点}:xxxx
\end{enumerate}
\end{frame}	




\subsection{数值模拟}





\section{序加实验下带区组效应寿命数据的统计分析}

\subsection{背景与动机}


\subsection{构建模型}



\subsection{统计推断}




\section*{参考文献}

\begin{frame}[allowframebreaks]{参考文献}
    \bibliography{ref}
    %\bibliographystyle{alpha}
    % 如果参考文献太多的话,可以像下面这样调整字体:
     \tiny\bibliographystyle{alpha}
\end{frame}

\begin{frame}
    \begin{center}
        {\Huge\calligra Thanks!}
    \end{center}
\end{frame}

\end{document}